\documentclass{article}
\usepackage{hyperref} 
\usepackage{fullpage}
\renewcommand{\labelitemi}{-}
\title{\textbf{Research Statement}}
\author{\textbf{Jonas von der Heyden}}
\date{}

\begin{document}

\maketitle

My research is concerned with the application of cryptography to real-world use cases. Thanks to the rapid development in both theoretical and applied cryptography, techniques that were considered to be of purely theoretical interest a decade ago are now at the threshold of practical feasibility or beyond. At the same time, there has only been a muted adoption of advanced cryptographic techniques such as multi-party computation (MPC) or fully homomorphic encryption (FHE) ``in the wild". I believe that in order to realize the untapped potential of these primitives, we need to assume an interdisciplinary lens. Instead of thinking in cryptographic primitives, it might be worthwhile to start with a problem from another discipline (e.g. electrical engineering or history) and investigate whether it can be solved by means from our cryptographic toolkit. By shifting my perspective to that of the problem domain—or, ideally, of the end user—I aim to unlock the power of advanced cryptography for impact in the real world.

\subsection*{Past and current research}
The main thrust of my doctoral dissertation is the design and implementation of advanced cryptographic protocols for applications involving resource-constrained (IoT) devices, such as passports and smart meters. Among others, my past research includes the following:
\begin{description}
    \item[Post-quantum authentication of passports] The \textit{Extended Access Control} (EAC) protocol is a cryptographic standard mandated by ICAO for the authentication of passports and the protection of sensitive data such as fingerprints. The protocol assumes the hardness of the discrete log, and since passports are typically valid for 10 years, the adoption of post-quantum secure primitives is an urgent objective. In our work \cite{FischlinHMMWB23}, we show how to replace the Diffie-Hellman key exchange with post-quantum secure KEMs while maintaining compatibility with the standard. We also give an efficient implementation of the protocol for passport-embedded smart cards, showing the practical feasibility of lattice-based KEMs in real-world applications. I designed the novel protocol, contributed to its implementation, and wrote most of the manuscript.
    \item[Privacy-preserving smart grids] Smart grids optimize our energy distribution, and help to alleviate climate change by allowing for the efficient integration of renewables into the electricity grid. Unfortunately they also require fine-grained usage measurements, which leak sensitive private information such as household occupancy, daily routine and even which movies are being watched. In our work \cite{HeydenSBAZJD25} we use MPC to implement one of the most complex algorithms used in smart grids, \textit{power flow analysis}, in a privacy-preserving manner such that no household has to reveal sensitive information. In the implementation, we overcome two main challenges. Firstly, we give an efficient MPC implementation for resource-constrained smart meters. Secondly, we provide optimizations that reduce the number of required communication rounds and thereby mitigate the effects of high-latency connections between smart meters. My contributions include the design and implementation of the entire privacy-preserving MPC solution and the authoring of all cryptography-related sections of the paper.
    \item[Revisiting rerandomizable garbling schemes] Motivated by our experiences in the smart grid project, we explore techniques that ease the adoption of advanced MPC protocols in real-world settings, namely outsourced MPC (addressing performance constraints) with constant round complexity (addressing network latency). The YOSO-like MPC protocol SCALES \cite{AcharyaHKP22}, which is based on rerandomizable garbling schemes (RGS), offers outsourcing under a dishonest majority and with constant-round complexity. Unfortunately, due to its use of BHHO \cite{BonehHHO08} as a building block, it is extremely inefficient. In our work (accepted at Crypto ‘25), we replace BHHO by a novel key-and-message homomorphic encryption (KMHE) scheme and improve space and runtime complexity of RGS by four to five orders of magnitude, making SCALES practically feasible for simple circuits. Moreover, we are currently working on a lattice-based RGS implementation. My primary contributions include participating in the ideation of the novel KMHE scheme, validating security proofs, conducting performance benchmarks, and authoring a comprehensive technical overview.
\end{description}
 \subsection*{Future objectives}
In my future research, I plan to continue working on the utilization of advanced cryptography to solve real-world and interdisciplinary problems. Among others, this entails finding innovative solutions to mitigate limitations around high-latency networks and resource-constrained devices, e.g. on IoT devices. Potential projects include:
\begin{description}
    \item[Privacy-preserving decentralized control systems for smart grids] Decentralized control for smart grids \cite{KILTHAU2025124606} enables a network of prosumers to manage their electricity grid in a distributed manner without centralized coordination. Beyond the intrinsic appeal of greater autonomy, such an architecture increases resilience and can lower costs by qualifying participants for take‑or‑pay tariffs. As in smart grids with centralized control, exchanging fine-grained usage measurements would reveal sensitive information. Therefore, implementing decentralized control in a privacy-preserving manner is an important research problem.
    \item[Privacy-preserving learning on IoT devices] Wearables and other IoT devices often use and train machine learning models. While privacy-preserving neural networks are mostly not practical for use on resource-constrained devices, there exist somewhat efficient privacy-preserving solutions for learning problems such as k-means, logistic- and linear regression. Moreover, simplified models with targeted privacy mechanisms might be a practical workaround to deal with limitations of IoT devices. For example, Bos et al. \cite{BosCIV17} show that privacy-preserving forecasting based on the group method of data handling can be efficient enough for use on smart meters. Considering the sensitivity of the information generated by (potentially health-related) wearables, efficient privacy-preserving learning algorithms on resource-constrained IoT devices with intermittent connectivity are an important research problem.
    \item[Study of historical ciphers] While virtually all historical encryption schemes, such as homophonic substitution ciphers, are considered to be broken, the actual decryption of historical secret letters remains non-trivial. This is mostly because we do not have enough encrypted material to extract sufficient statistical information about the ciphertext, but it might also be due to the use of codes (``nomenclatures") or encryption errors. Therefore, interdisciplinary cooperation is essential: cryptographers contribute semi-automated tools for cryptanalysis, while historians provide the necessary contextual knowledge and proficiency  in historic languages to aid decryption. Working with historians at the University of Wuppertal, I supervised three computer‑science bachelor theses that culminated in the decipherment of a Renaissance‑era letter from Cosimo~de'~Medici to Francesco~Sforza.
 Decipherments of such historical secret letters
 %, for example from Mary Stuart \cite{LasryBT23} or Charles~V \cite{PierrotDGZ23}, have 
 reveal diplomatic secrets of the medieval and Renaissance periods, offering a glimpse into the private correspondence of important historic figures and prompting reassessments in historical research.
    \item[Hybrid blind signatures] The European Central Bank (ECB) is currently investigating the feasibility of a ``digital Euro''. A key requirement is that the digital Euro should be anonymous, i.e. it should not be possible to link transactions to users. Traditionally, this is achieved by using blind signatures, which allow a user to obtain a signature on a message (a banknote) without revealing the message to the signer (the bank) \cite{Chaum82}. However there are currently no known constructions for hybrid blid signatures, which combine classical and post-quantum security. To safeguard against both quantum adversaries and potential weaknesses in post-quantum algorithms,  it is crucial to develop efficient constructions for hybrid blind signatures or generic combiners.

    % \item[Censorship-resistance via YOSO (“you only speak once”) protocols] Most technological solutions to protect privacy against nation-state adversaries (such as E2EE instant-messaging) are run by a centralized entity (e.g. the Signal foundation) and could be disabled by a dictator who corrupts the entity, or simply blocks its IP-address(es). To protect against this threat, mechanisms of censorship-resistance are required. Firstly, the application code needs to be open-source. Secondly, we need techniques to enable unblockable (potentially P2P) communication based on weak assumptions (e.g. continued existence of the internet). In YOSO protocols, participants appear spontaneously, speak only once and then disappear, hence blocking their IP-address is futile. A line of research (\cite{GentryHKMNRY21,AcharyaHKP22}, our submission to Crypto ‘25) has shown that YOSO-MPC is practical for simple circuits. Another feature that makes YOSO attractive for censorship-resistance is that it does not require pre-existing secure point-to-point channels. Therefore it is natural to ask: Is it practically feasible to run censorship-resistant privacy-preserving applications by replacing a centralized server with ephemeral participants using YOSO-like protocols? 
\end{description}
\bibliographystyle{alpha}
\bibliography{references}
\end{document}
